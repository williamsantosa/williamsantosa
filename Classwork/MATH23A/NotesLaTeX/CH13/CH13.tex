\documentclass[12pt, letterpaper]{article}

\usepackage[margin=1.25in]{geometry}
\usepackage{amsmath, amssymb}
\usepackage{graphicx}
\usepackage{authblk}
\usepackage{indentfirst}
\usepackage{blindtext}
\usepackage{hyperref}
\graphicspath{{./Images/}}
\setlength{\parindent}{12pt}
\setlength{\parskip}{1em}

\title{\vspace{-0.4cm}MATH23A Ch.12 Notes}
\author{\vspace{-0.4cm}William Santosa}
\date{\vspace{-0.4cm}Winter 2022 Quarter}

\begin{document}

\maketitle

\section{Introduction}

Geometry is cool and we better learn it cause it will be important for later classes.

\section{The Geometry of Real-Valued Functions}

Let \(f\) be a function whose domain is a subset \(A\) of \(\mathbb{R}^n\) and with a range contained in \(\mathbb{R}^m\). That means to each \(x = (x_1, ..., x_n) \in A\), \(f\) assigns a value \(f(x)\), an \(m\)-tuple in \(\mathbb{R}^m\). These functions are called \textit{vector-valued functions} if \(m > 1\) and \textit{scalar-valued functions} if \(m = 1\). 
\begin{enumerate}
    \item Scalar-valued: \(f(x, y, z) \rightarrowtail (x^2 + y^2 + z^2)^{-3/2}\), since it outputs a single value, not a 2 tuple with 2 or more elements.
    \begin{itemize}
        \item Could be written as \(f: (x, y, z) \rightarrowtail (x^2+y^2+z^2)^{-3/2}\) since \(\mathbb{R}^3 \rightarrow \mathbb{R}^1\).
    \end{itemize}
    \item Vector-valued: \(g(x) = g(x_1, x_2, x_3, x_4, x_5, x_6) = (x_1x_2x_3x_4x_5x_6, \sqrt{x_1^2+x_6^2})\) since \(\mathbb{R}^6 \rightarrow \mathbb{R}^2\).
\end{enumerate}

We usually use the notation \((x,y,z)\) rather than \(x_1,x_2,x_3\) for tuples in \(\mathbb{R}^3\).
The notation \(x \rightarrowtail f(x)\) is useful for the indicating the value to which a point \(x \in \mathbb{R}^n\) is sent. That means we write \(f: A \subset \mathbb{R}^n \rightarrow \mathbb{R}^m\) to signify that A is the domain of \(f\text{(a subset of }\mathbb{R}^n)\) and the range is contained in \(\mathbb{R}^m\). We also use the expression \(f \text{ maps } A \text{ into } \mathbb{R}^m\). Such functions \(f\) are called \textit{functions of several variables} if \(A \subset \mathbb{R}^n,n>1\).

When \(f: U \subset \mathbb{R}^n \rightarrow \mathbb{R}\), we say that \(f\) is a \textit{real-valued function of n variables with domain U}.

Graph of a Function is defined by: Let \(f: U \subset \mathbb{R}^n \rightarrow \mathbb{R}\). Define the \textit{graph} of \(f\) to be the subset of \(\mathbb{R}^{n+1}\) consisting of all the points 
\[ 
(x_1,...,x_n,f(x_1,...,x_n))
\] 
in \(\mathbb{R}^{n+1}\) for \((x_1,...,x_n)\) in \(U\). In symbols,
\[ 
\text{graph }f = \{(x_1,...,x_n,f(x_1,...,x_n)) \in \mathbb{R}^{n+1} \mid (x_1,...,x_2) \in U\}
\] 

If \(n = 1\), the graph is a curve in \(\mathbb{R}^2\). For \(n = 2\), it is a surface in \(\mathbb{R}^3\).

A level set is the set of \((x,y,z)\) where \(f(x,y,z) = c\), where c is a constant. Levels sets are useful for understanding functions of two variables, \(f(x,y)\), in which case we speak of \textit{level curves} or \textit{level contours}.

Level curves/surfaces are defined by: Let \(f: \mathbb{R}^n \rightarrow R\) and let \(c \in \mathbb{R}\). Then the \textit{level set of value} c is defined to be the set of those points \(x \in U\) at which \(f(x) = c\). If \(n = 2\), we speak of a \textit{level curve} (of value c); and if \(n = 3\), we speak of a \textit{level surface}. In symbols, the level set of value c is written 
\[ 
\{x \in U \mid f(x) = c\} \subset \mathbb{R}^n
\] 
Note that the level set is always in the domain space.

A \textit{section} of the graph of \(f\) is the intersection of the graph and a (vertical) plane.

\section{Limits and Continuity}

We learn about open sets, limits, and continuity in this section. Open sets are needed to learn limits, limits are needed to learn continuity and differentiability.

Open Set. Let \(U \subset \mathbb{R}^n\). We call \(U\) an \textit{open set} when for every point \(x_0\) in \(U\) there exists some \(r>0\) such that \(D_r(x_0)\) is contained within \(U\); symbolically, we write \(D_r(x_0) \subset U\).

For each \(x_0 \in \mathbb{R}^n, D_r(x_0)\) is an open set.
\end{document}