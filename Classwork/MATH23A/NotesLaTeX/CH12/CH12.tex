\documentclass[12pt, letterpaper]{article}

\usepackage[margin=1.25in]{geometry}
\usepackage{amsmath, amssymb}
\usepackage{graphicx}
\usepackage{authblk}
\usepackage{indentfirst}
\usepackage{blindtext}
\usepackage{hyperref}
\graphicspath{{./Images/}}
\setlength{\parindent}{12pt}
\setlength{\parskip}{1em}

\title{\vspace{-0.4cm}MATH23A Ch.12 Notes}
\author{\vspace{-0.4cm}William Santosa}
\date{\vspace{-0.4cm}Winter 2022 Quarter}

\begin{document}

\maketitle

\section{Introduction}

Welcome to Chapter 12. We will learn about Paths and Curves in space, which in turn allow us to model different situations in real life. 

\section{Paths and Curves}

We usually think of a curve as line drawn on a paper, yet, it is useful to think of a curve \textit{C} mathematically as a set of values of a function that maps an interval of real numbers into the plane or space. Thus, we shall call such a map a \textit{path}. 
We usually denote a path by \textit{c}. The image of \textit{C} of the path then corresponds to the curve we see on paper. Often, we write \textit{t} for the independent varaible and imagine it to be \textit{time}, so that \(c(t)\) is the position at time \textit{t} of a moving particle, which \textit{traces out} a curve as \textit{t} varies. Additionally, we say \textit{parametrizes C}. Thus, we should distinguish between \(c(t)\) as a \textit{point} in space and as a \textit{vector} based at the origin.

A path in \(\mathbb{R}^n\) is a map \(c: [a, b] \rightarrow \mathbb{R}^n\); it is a \textit{path in the plane} if n = 2 and a \textit{path in space} if n = 3. The collection \textit{C} of points \(c(t)\) as \textit{t} varies in [a, b] is called a \textit{curve}, and c(a) and c(b) are its \textit{endpoints}. The path c is said to \textit{parametrize} the curve C. We also say c(t) \textit{traces out C} as t varies.

If c is a path in \(\mathbb{R}^3\), we can write \(c(t) = (x(t), y(t), z(t))\), and we call \(x(t), y(t), \text{ and } z(t)\) the \textit{component functions} of c. We form components functions similarly in \(\mathbb{R}^2\) or, generally, in \(\mathbb{R}^n\). We also consider paths whose domain is the whole real line as in the next example.

The curve when the wheel rolls on a circle is called an \textit{epicycle}. When the wheel is outside the circle and the point is on the rim, the curve is called an \textit{epicycloid}. When the wheel is inside the circle, it is a \textit{hypocycloid}.

Projections are a way to visualize space curves. The projection of a path \(c(t) = (x(t), y(t), z(t))\) onto the xy-plane is \(p(t) = (x(t), y(t), 0)\). Basically, set the variable that is not part of the plane to 0. That means the projections onto the yz- and xz-planes are the paths \((0, y(t), z(t)) \text{ and }(x(t), 0, z(t))\), respectively.

To parametrize a curve obtained as the intersection of two surfaces,
\begin{enumerate}
    \item Solve the given equations for y and z in terms of x. First, solve for y.
    \item Substitute y into the other equation and solve for x.
    \item Set x = t.
    \item Then, the new vector(s) are the functions to parametrize the entire curve.
\end{enumerate}
Alternatively,
\begin{enumerate}
    \item Find the trigonometric parametrization of one of the equations. 
    \item Plug values into the second equation and solve for z.
    \item Plug into c(t) = (x(t), y(t), z(t)).
\end{enumerate}

Path c(t) approaches the limit u (a vector) as t approaches \(t_0\) if \(\text{lim}_{t \rightarrow t_0}||c(t) - u|| = 0\). In this case, we write 
\[
\text{lim}_{t \rightarrow t_0}c(t) = u    
\]
That means a path \(c(t) = (x(t), y(t), z(t))\) approaches a limit as \(t \rightarrow t_0\) if and only if each component approaches a limit, and in this case,
\[
\text{lim}_{t \rightarrow t_0}c(t) = (\text{lim}_{t \rightarrow t_0}x(t), \text{lim}_{t \rightarrow t_0}y(t), \text{lim}_{t \rightarrow t_0}z(t))    
\]
Thus, c(t) is differentiable at t if the following limit exists.
\[
c^{'}(t) = \frac{d}{dt}c(t) = \text{lim}_{h \rightarrow 0}\frac{c(t + h) - c(t)}{h}    
\]
Basically, just find the derivative of each component within c(t) to differentiate it. Remember that the derivative is defined as 
\[
f^{'}(x) = \frac{d}{dx}f(x) = \frac{d}{dx}[x^n] = nx^{n - 1}
\]

Thinking of c(t) as the curve traced out by a particle and t as time, it is reasonable to define the velocity vector as follows. If c is a path and it is differentiable, we say 
c is a \textit{differentiable path}. The velocity of c at time t is defined by 
\[
c^{'}(t) = \text{limt}_{h \rightarrow 0}\frac{c(t+h) - c(t)}{h}    
\]
\begin{enumerate}
    \item The vector \(c^{'}(t)\) is usually drawn with its tail at \(c(t)\).
    \item The \textit{speed} of the path c(t) is \(s = ||c^{'}(t)||\), which is the length of the velocity vector.
    \item If \(c(t) = (x(t), y(t))\) in \(\mathbb{R}^2\), then \[c^{'}(t) = (x^{'}(t), y^{'}(t)) = x^{'}(t)i + y^{'}(t)j\]
    \item Similarly, if \(c(t) = (x(t), y(t), z(t))\) in \(\mathbb{R}^3\), then \[c^{'}(t) = (x^{'}(t), y^{'}(t), z^{'}(t)) = x^{'}(t)i + y^{'}(t)j + z^{'}(t)k\]
\end{enumerate} 

The velocity \(c^{'}(t)\) is a vector \textit{tangent} to the path c(t) at time t. If C is a curve traced out by c and if \(c^{'}(t)\) is not equal to 0, then \(c^{'}(t)\) is a vector tangent to the curve C at the point c(t).
\begin{enumerate}
    \item Find the derivative of the path to obtain the tangent vector in terms of t. 
    \item Place in the form \(l(t) = c(t_0) + (t - t_0)(c^{'}(t_0))\) where 
    \begin{enumerate}
        \item \(c(t) = \) the path.
        \item \(t_0 = \) the value of t at a certain point.
        \item \(c^{'}(t) = \) the tangent vector.
    \end{enumerate}
    \item Plug in t to obtain the equation of the tangent line l(t) to the path.
\end{enumerate}

\section{Acceleration}

Same thing as before, except we now learn about more topics including acceleration.

A path in \(\mathbb{R}^n\) is a map c of \(\mathbb{R}\) in \(\mathbb{R}\) to \(\mathbb{R}^n\). The derivative at each time t is a vector with n components. Baiscally, the derivative of \(c(t)\), \(c^{'}(t)\) is given by 
\[
c^{'}(t) = (dx_1/dt_1, ..., dx_n/dt) \text{     or     } c^{'}(t) = (x^{'}_{1}(t), ..., x^{'}_{n}(t))
\]
Also recall that if c represents the path of a moving particle, the velocity vector is
\[
v = c^{'}(t)    
\]
and its speed is \(s = ||v||\).

Let b(t) and c(t) be differentiable paths in \(\mathbb{R}^3\) and p(t) and q(t) be differentiable scalar functions. The following are the differentation rules.
\begin{itemize}
    \item Sum Rule: \(\frac{d}{dx}[b(t) + c(t)] = b^{'}(t) + c^{'}(t)\)
    \item Scalar Multiplication Rule: \(\frac{d}{dt}[p(t)c(t)] = p^{'}(t)c(t) + p(t)c^{'}(t)\)
    \item Dot Product Rule: \(\frac{d}{dt}[b(t) \cdot c(t)] = b^{'}(t) \cdot c(t) + b(t) \cdot c^{'}(t)\)
    \item Cross Product Rule: \(\frac{d}{dt}[b(t) \times c(t)] = b^{'}(t) \times c(t) + b(t) \times c^{'}(t)\)
    \item Chain Rule: \(\frac{d}{dt}[c(q(t))] = q^{'}(t)c^{'}(q(t))\)
\end{itemize}

For a path describing uniform rectilinear motion, the velocity vector is constant. Generally, the velocity vector is a vector fuunction \(v = c^{'}(t)\) that depends on t. The derivative \(a = dv/dt = c^{''}(t)\), if it exists, is called the acceleration of the curve. If a curve is \((x(t), y(t), z(t))\), then the acceleration at time t is given by 
\[
a(t) = x^{''}(t)i + y^{''}(t)j + z^{''}(t)k    
\]

\section{Arc Length}

Sometimes we need to find the length of a path c(t). How do we do this? Well, simply, with represent to th etime over the interval \([t_0, t_1]\) of travel time, the length of the path, also known as \textit{arc length}, in 
\[
L(c) = \int_{t_0}^{t_1}||c^{'}(t)||dt.
\]
Alternatively, the length of the path \(c(t) = (x(t), y(t), z(t))\) for \(t_0 \leq t \leq t_1\) is 
\[
L(c) = \int_{t_0}^{t_1}\sqrt{[x^{'}(t)]^2 + [y^{'}(t)]^2 + [z^{'}(t)]^2}    
\]
An \textit{infinitesimal displacement} of a particle following a path \(c(t) = x(t)i + y(t)j + z(t)k\) in 
\[
ds = dxi + dyj + dzk = (\frac{dx}{dt}i + \frac{dy}{dt}j + \frac{dz}{dt}k)dt
\]
and its length
\[
ds = \sqrt{dx^2 + dy^2 +dz^2} = \sqrt{(\frac{dx}{dt})^2 + (\frac{dy}{dt})^2 + (\frac{dz}{dt})^2} dt 
\]
To more easily remember the arc length formula, we can just remember it as 
\[
\text{arc length} = \int_{t_0}^{t_1}ds
\]
\end{document}